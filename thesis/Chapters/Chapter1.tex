% Chapter 1

\chapter{Introducción} % Main chapter title

\label{Chapter1} % For referencing the chapter elsewhere, use \ref{Chapter1} 

%----------------------------------------------------------------------------------------

% Define some commands to keep the formatting separated from the content 
\newcommand{\keyword}[1]{\textbf{#1}}
\newcommand{\tabhead}[1]{\textbf{#1}}
\newcommand{\code}[1]{\texttt{#1}}
\newcommand{\file}[1]{\texttt{\bfseries#1}}
\newcommand{\option}[1]{\texttt{\itshape#1}}

%----------------------------------------------------------------------------------------

\section{Notas sobre la redacción}

Aparte de darle un repaso general a la escritura y añadir algunos esquemas para que queden más claras algunas arquitecturas e ideas, las partes que me faltan son las siguientes (lo dejo aquí para recordarme que lo tengo que hacer):

\begin{itemize}
    \item{Intruducción}
    \item{Explicación de conceptos básicos de redes neuronales, que son, algoritmos de optimización (gradiente del descenso), backpropagación, etc}
    \item{Pasar a aquí el análisis exploratorio del dataset (histogramas por tamaños, colores, etc)}
    \item{Terminar documentación del modelo de multi output (bounding boxes)}
    \item{Explicar los obstáculos de usar otros modelos convolucionales preentrenados (inception y resnet sobretodo)}
    \item{Terminar documentación del modelo de ensemble}
    \item{Conclusiones}
    \item{Apéndice explicando el uso de Keras para el deep learning}
\end{itemize}




