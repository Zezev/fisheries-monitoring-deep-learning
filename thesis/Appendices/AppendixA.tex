% Appendix A

\chapter{Infraestructura} 
\label{ap:hardware}

\section{Requisitos}

Al principio de este trabajo se comentó que era necesario un enfoque de desarrollo rápido para probar diferentes ideas y encontrar rápidamente los errores y mejoras. Los algoritmos de aprendizaje sobre redes convolucinoales son tradicionalmente costosos en cómputo, por lo que es una dificultad a superar si queremos una evolución rápida de la solución.

\section{Hardware}

\textit{Keras} permite aumentar la velocidad de entrenamiento usando el procesador gráfico del sistema, siempre que este sea compatible con CUDA (). Para este problema se han usado dos equipos diferentes, ambos compatibles con CUDA.

Para la mayoría de entrenamientos y experimentos se ha usado un ordenador de las siguientes características:

\begin{itemize}
    \item{Procesador: Intel i3, 2.7 GHz}
    \item{Procesador gráfico: Nvidia GeForce GTX 950, 2GB de memoria}
    \item{16 GB de RAM}
\end{itemize}

Este equipo era suficiente para la mayoría de entrenamientos. Sin embargo cuando era necesario entrenar un conjunto de datos más grande, como la segunda fase de la clasificación, con 13000 ejemplos, el equipo tardaba demasiado en entrenar algunos modelos.

Por eso se optó por buscar una solución en la nuben que permitiese alquilar una potencia gráfica superior durante tiempo limitado. Al final se ha optado por las máquinas virtuales de \textit{Amazon AWS}, que permite contratar máquinas por horas. Una de ellas, identificada \textit{p2.xlarge}, ofrece las siguientes características:

\begin{itemize}
    \item{Procesador: Intel i7, 3.5 GHz, 4 núcleos}
    \item{Procesador gráfico: Nvidia Tesla K80}
    \item{61 GB de RAM}
\end{itemize}

Este equipo, con un precio de 0.80 € la hora, ofrece potencia más que suficiente para entrenar grandes cantidades de datos rápidamente.

